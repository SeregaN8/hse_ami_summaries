\documentclass[12pt]{article}

\usepackage{cmap}
\usepackage[T2A]{fontenc}
\usepackage[utf8]{inputenc}
\usepackage[english, russian]{babel}
\usepackage{amsthm,amsmath,amssymb,tabto}
\usepackage{hologo}
\newtheorem{definition}{Definition}
\newcommand{\floor}[1]{\left\lfloor #1 \right\rfloor}
\newcommand{\ceil}[1]{\left\lceil #1 \right\rceil}

\usepackage{listings}
\usepackage{color}
\usepackage{xcolor}

\title{Конспект по алгебре 2 семестр}


\begin{document}
\maketitle
\section{Векторные пространства}
\begin{definition}
K - поле (скаляры - чиселки), векторное пространство над K - это тройка (V + $\cdot$), где плюсик и точечка это бинарные операции. Сложение $V + V \Rightarrow V$, умножение (на скаляр) $K \cdot V \Rightarrow V$
\end{definition}

Аксиомы: $(V +)$ - абелева группа: \begin{enumerate}
    \item $\overline{0}$ - нейтральный элемент 
    \item $-v$ обратный по сложению
    \item Коммутативность по сложению
    \item Ассоциативность по сложению
    \item $(k_1 * k_2) * v = k_1 * (k_2 * v)$ - согласованность операций (тут звёздочки немного разное умножение - сначала перемножение скаляров, потом домножение на скаляр)
    \item $k (v_1 + v_2) = kv_1 + k v_2$
    \item $(k_1 + k_2) v = k_1v + k_2v$
    \item $1_K * v_1 = v_1$ (тут и в дальнейшем под $0_V$, $1_K$ и в прочих подобных записях имеется в виду 0 или 1 из соответствующего множества)
\end{enumerate} 
Для аксиом $\forall k1,\  k2 \in K, \ , \forall v_1, \ v_2 \in V$
\\

Замечания:
\begin{enumerate}
    \item  $0_K * v = 0_V$
    \item $(-1) v = -v$ 
\end{enumerate}
Доказательства замечаний: 
\begin{enumerate}
    \item $0_K * v = (0_K + 0_K) * v = 0_K * v + 0_K * v$, и т.к. $V$ - абелева, то вычтем обратное к $0_K * v$ из обеих частей и получим $0_K * v = 0_V$
    \item $(-1) v + v = (-1 + 1)v = 0_K*v = 0_V$, значит $(-1)v$ - обратный к $v$
\end{enumerate}
\\

\textbf{Пример:}  
\begin{enumerate}
    \item Вектора на плоскости (классы эквивалентности направленных отрезков) - можем каждый вектор отложить от начала координат и на отложенных векторах определить операции
    \item Арифметическое векторное пространство - $K^n = \{ \begin{pmatrix}
    K_1 \\
    K_2\\
    ... \\
    K_n
  \end{pmatrix} |\  \forall K_i \in K\}$ на такой красоте можем определить умножение, и домножение на скаляр, для них работают аксиомы (не верите - проверите)
  Кста, еще это в.п. называется $n$-мерным арифметическим пространством столбцов, (но использоваться это название будет редко либо не будет)
  \item $ ^nK = \{ K_1, K_2, ..., K_n \}$ (пространство строк)
  
  $K^n$ и $ ^nK$ - изоморфны, $\exists$ биективный гомоморфизм
\end{enumerate} 

 



\begin{definition}
    U, V - векторные пространства над K, $f$ : $U \rightarrow V$ - гомоморфизм (линейное отображение) - значит:
    \begin{enumerate}
        \item f(a + b) = f(a) + f(b)
        \item f(ka) = kf(a) 
        \item отображения это геометрическая конструкция
    \end{enumerate}
    ($ \forall$ $ a, b \in U$, $f(a), f(b) \in V$, $k \in K$)
\end{definition} 


\textbf{Больше примеров Богу примеров:}
\begin{enumerate}
    \item $K[x]$ - векторное пространство над K
    \item $K[x^n]=\{f\ |\ deg\ f\leq n\}$ - векторное пространство над K
    \item $K[[x]]$ - векторное пространство над K
    \item $R$ - коммутативное кольцо, $K \subset R$ - поле(подкольцо), тогда $R$ - в.п. над $K$
    \item $\mathbb{C}$ - в.п над $\mathbb{R}$
    \item $\mathbb{R}$ - в.п над $\mathbb{Q}$
\end{enumerate}

\textbf{Ещёёёё примеры} \begin{enumerate}
    \item M - множество, $V := {f : M \rightarrow K}$ - векторное пространство над $K$, потому что $(f_1+f_2)(m):=f_1(m)+f_2(m)$, $(k\cdot f)(m):=k\cdot f(m)$ (отобразили элемент из $M$ в $K$, получили элемент $K$, домножили на другой элемент из $K$)
    \item $func(\mathbb{R}, \mathbb{R})$ - векторное пространство, $func_c(\mathbb{R}, \mathbb{R})$ - тоже векторное над $\mathbb{R}$, но непрерывное
    \item $(a_1, a_2, ..., a_n) \in R$ будем называть Фибоначчиевой, если $\forall i > 2 \ a_i = a_{i - 1} + a_{i - 2}$. Для них справедливо что $a_i + b_i$ и $ka_i$ - тоже фибоначчиевы, значит в.п.
    \item $V$ - множество подмножеств, $K$ - множество вычетов по модулю 2 ($\mathbb{Z}{/2\mathbb{Z}}$), $+_V$ определим как симметрическую разность ($X+Y:=(X\cup Y)\setminus (X\cap Y)$), умножение как $0*X = \emptyset,\ 1*X = X$, получили векторное пространство. \textbf{Замечание} важно что модуль 2, потому что $(1 + 1) X = X + X = 0$ в $K$, тогда $char(k) = 2$
\end{enumerate}
\\

\textbf{Как ввести координаты в наши примеры:}
\begin{enumerate}
    \item $K[x]_n = \{a_0, a_1, .. , a_n\} = (a_0, a_1, ..., a_n)$ $K[x]_n \cong K^{n + 1}$
    \item $\mathbb{C}$ над $\mathbb{R}$ $z = a + bi$, $\mathbb{C} \cong \mathbb{R}^2$
    \item Пример плохих координат: $z \rightarrow (r, \phi)$ - не согласуется с операциями
    \item $Func(\mathbb{N}, \mathbb{R})$ - $\mathbb{R}^{n}$ тут n типа $\infty$
    \item любая фибоанчиева кодируется первыми двумя номерами, а значит $(a_1, a_2) \cong \mathbb{R}^2$ 
    \item Подмножества это бинарная строка $N \subset M$, $N = (\epsilon_1, ..., \epsilon_n), \ \forall \epsilon_i = 
 \begin{cases}
   1 & a_i \in M\\
   0 & a_i \notin M
 \end{cases}$
\end{enumerate} 



\section{Базис и размерность} 
\begin{definition}
    $v_1, .. v_n \in V$ - векторное пространство над $K$, a-шки это скаляры ($\forall a_i \in K$), тогда $a_1v_1 + a_2v_2 + ... + a_nv_n$ - линейная комбинация $v_1, ..., v_n$ с коэффициентами $a_i$. 
\end{definition}


\textbf{Замечание} множество линейных комбинаций ($v_1, v_2, ..., v_n \in V$) замкнуто относительно (+ $\cdot$), т.е. является векторным пространством. Это линейная оболочка $v_1, v_2, ..., v_n = \langle v_1, ..., v_n \rangle$, $\sum a_iv_i+\sum b_iv_i=\sum (a_i+b_i)v_i;\ k\cdot \sum a_iv_i=\sum (ka_i)\cdot v_i$ \\


\textbf{Замечание 2} всё тоже можно представить для бесконечных систем, но тогда опасно определять линейную комбинацию (для этого говорим что конечное число не нулей). Мы в основном будем говорить про конечные системы
\\

\textbf{Примеры} Для одного вектора линейная оболочка это прямая, для двух это плоскость (если они не коллинеарны) 


\begin{definition}
    Условия линейной независимости системы векторов $v_1, v_2, ..., v_n \in V$:
    \begin{enumerate}
        \item  Никакой вектор не выражается через остальные $\forall i\ v_i\neq \sum_{j\neq i} {a_jv_j}$ ($\forall a_i$)
        (прикол - сумма по пустому множеству это $O_V$ , один нулевой вектор линейно зависим)
        \item Никакая нетривиальная линейная комбинация не равна нулю $\sum_{j} {a_jv_j}=0\Rightarrow \forall i\ a_i=0$
    \end{enumerate}

\end{definition}
 

\textbf{Докажем равносильность} - очень очев: 

$2 \Rightarrow 1$ пусть $v_i = \sum_{\forall j \neq i} {a_jv_j}$, тогда перенесем $v_i$ в сумму и получим нетривиальную линейную комбинацию с нулевой суммой (нетривиальна потому что при $v_i$ коэффициент $-1$), противоречие

$1 \Rightarrow 2$ пусть существует нетривиальная комбинация, тогда перенесем какой-нибудь $v_i$ с ненулевым коэффициентом в другую часть равенства и поделим обе части равенства на коэффициент при $v_i$, выразили $v_i$ через остальные, противоречие.

(тут важно что k это поле, потому что делим на $a_i$ из поля и можем так делать) 
В дальнейшем может использоваться сокращение л.н.с. - линейно независимая система и аналогично л.з. - линейно зависимая.

\textbf{Замечание} Можно сделать такое для ассоциативного кольца (с 8 аксиомами) и получим $(V + *)$ - называется R-модулем (или модуль над $R$)

\begin{definition}
    V - векторное пространство над K, $v_1, v_2, ..., v_n \in V$ - система векторов, тогда она называется порождающей если $\langle\{v_i\} \rangle =V$
\end{definition} 
\\

\textbf{Замечание} - пусть $M \subset V$ - система векторов, если M - л.н.с, то $N \subset M$ - л.н.с, и если M - порождающая, и $M \subset N$, то $N$ - порождающая
\\

\begin{definition}
    $\{v_i\} \subset V$, $V$ - векторное пространство, $\{v_i\}$ - базис, если выполнены 4 равносильных условия
    \begin{enumerate}
        \item $\{v_i\}$ - л.н.с. и порождающая
        \item $\{v_i\}$ - максимальная л.н.с. (добавим хоть 1 любой вектор из $V$ - и станет зависимой)
        \item $\{v_i\}$ - минимальная порождающая (уберем хоть 1 вектор - перестанем быть порождающей)
        \item $\forall v \in V$ - представляется единственным образом как линейная комбинация $\{v_i\}$ 
    \end{enumerate}
    p.s. в общем-то примерно так же работает и на бесконечных системах, но стоит помнить про тонкости с линейной комбинацией, и опять же, мы с таким работать почти не будем.

\end{definition} 

\textbf{Доказательства:} План: $3 \Leftrightarrow 1 \Leftrightarrow 2$ и где-то добавить 4..

$1 \rightarrow 2$ добавим любой вектор, он выражается через предыдущие т.к. система порождающая, следовательно перестали быть л.н.с., ч.т.д. (т.е. текущая система - максимальная л.н.с.)

$2 \rightarrow 1$ пусть мы не порождающая система, тогда мы не можем выразить из текущих векторов какой-то вектор $u \in V$, добавим его в систему, получим все еще л.н.с., противоречие с максимальностью

$1 \rightarrow 4$ Порождающая $\Rightarrow$ каждый вектор можно выразить хотя бы одним способом. Пусть какой-то можно выразить 2-мя разными способами, тогда его разность с самим собой (т.е. $O_V$) можно представить как линейную комбинацию с нетривиальным набором коэффициентов, противоречие с л.н.с.

$4 \rightarrow 1$ любой вектор представляется хотя бы 1 способом $\Rightarrow$ система порождающая. Л.н.с.: пусть мы линейно зависимы, тогда воспользуемся первой трактовкой л.н.с.: какой-то вектор из нашей системы выражается через остальные. Тогда противоречие с единственностью: с одной стороны, этот вектор выражается просто как он сам, с другой стороны, как линейная комбинация остальных векторов из системы.

$3 \Rightarrow 1$ Порождаемость уже есть, пусть мы л.з. Тогда уберем какой-нибудь вектор, выражающийся через остальные (остались порождающей системой т.к. там где раньше участвовал удаленный вектор подставим его выражение через остальные вектора). Противоречие с минимальностью порождаемости

$1 \Rightarrow 3$ порождаемость есть, пусть не минимальная, тогда можно удалить какой-нибудь элемент, не потеряв порождаемость (т.е. этот элемент выражается через остальные). Противоречие с л.н.с.


Определение - V - пространство, ${v_i}$ - базис, для $\forall v \in V \exists !{a_i}$ - координаты в базисе;;; $v \rightarrow ^nK$ - изоморфизм
д-во - биективность корректность гомоморфность 

Базис - строка, координаты - столбец

Пример с трёхчленом для базиса $(1 , x, x^2)$, $(x^2 + 1, x^2 + x + 3, x^2 - x)$ 


Существование базиса - $V$ - называется конечномерным - если в $V$ существует конечная порождающая система. $V$ - является линейной оболочкой конечного числа векторов.
Лемма - из любой конечно мерной системы можно извлечь базис
Теорема - из любого конечно мерного пространства можно извлечь базис

Доказательство леммы - доказать, что если один вектор выражается через остальные, то его можно убрать, не потеряв порождаемость. Важно что система конечная, т.е. удаляем вектора пока не станем л.н.с. и в любом случае сделаем конечное число шагов.

Замечание (!!!) в любом пространстве есть базис, 

Лемма цорна(??)

Когда базис бесконечный - говорим что его нет (короче опять тонкости про бесконечномерные пространства, Антипов либо расскажет потом либо забьет)

определение - $V$ - векторное пространство(конечномерное), размерность $V$ $(dim(V))$ - количество векторов в его базисе (резонный вопрос: а правда ли что во всех базисах поровну элементов? ответим на него в следующей теореме)


Факт(теорема) - в любых двух базисах поровну элементов. Доказательство - следует из леммы.

Лемма О линейной зависимости линейных комбинаций  (лзлк)

$(u_1, .., u_n)$ - все $u_i$ - линейные комбинации из $(v_1, v_2, .. , v_m)$ и $n > m$, тогда ушки линейно зависимы - почему

доказательство - н.у.о. $n = m + 1$, потому что умеем убирать лишнее, индукция по $n$ - база - $n = 2$, $u_1 = a_1v_1, u_2 = a_2v_1$, если есть ноль - то очев зависимо, если нет, то второй выражается через первый

переход $n \rightarrow n + 1$ каждая $u$ выражается через $v$.., пробуем исключить последнюю $v_{n + 1}$, если смогли, тогда по предположению победа, если не смогли $(u_1 = ... + a_{n + 1}v_{n + 1})$ - есть такой - у которого не нуль тогда повычетаем с константами чтобы последнее умерло 
\end{document}
